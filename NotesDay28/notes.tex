\documentclass[12pt]{article}

\usepackage{import}
\usepackage{xifthen}
\usepackage{pdfpages}
\usepackage{transparent}

\usepackage{geometry}
\usepackage{graphicx}
\usepackage{subcaption}
\usepackage{float}
\usepackage{amsmath}
\usepackage{mathtools}
\usepackage{fontspec}
\setmainfont{Iosevka SS01}

\newcommand{\incfig}[1]{%
    \def\svgwidth{\columnwidth}
    \import{./figures/}{#1.pdf_tex}
}

% Title Page
\title{Calc III Notes Day 28}
\author{Joseph Brooksbank}

\begin{document}
\maketitle

\section*{Continued From last Class}
Estimating Inf series / integrals
\\
\fbox{\begin{minipage}{35em}
suppose you have some inf series $\sum_{n=1}^{\infty} x_n $ and it converges... but you don't know the value.
\end{minipage}}
\\
If you want to approximate $\sum_{n=1}^{\infty} x_n $ The best way to do it is just to add up a "bunch" of terms 
\\
Aka use the sum of the first \textbf{N} terms.
\\
\fbox{\begin{minipage}{15em}
                Question: How far off is $\sum_{n=1}^{N}x_n  $ from $\sum_{n=1}^{\infty} x_n $? 
\end{minipage}}
\subsection*{Type of question: Approximate $\sum_{n=1}^{\infty}\frac{1}{n^{3}}$ to within 0.01:}

\begin{align*}
        \sum_{n=1}^{N}x_n &= x_1 + x_1 +x_3...+x_N \\
        \sum_{n=1}^{\infty}x_n &= x_1 + x_2 + x_3 +... + x_N + x_{N+1}...
\end{align*}
This the sum to N is off by $x_{N+1} + x_{N+2}...$ 

\fbox{\begin{minipage}{15em}
                KEY::
                $\sum_{n=1}^{\infty}x_n < \int_{N}^{\infty}f(x)  dx $
\end{minipage}}
Basically, if we can get the integral to go to converge than we can squeeze the series. 
\\
\begin{figure}[ht]
    \centering
    \incfig{newfig}
    \caption{newfig}
    \label{fig:newfig}
\end{figure}
Set up 
\begin{align*}
        \int_{N}^{\infty} \frac{1}{x^{3}} dx &= \lim_{t\to\infty} \int_{N}^{t} x^{-3} dx \\
                                             &= \lim_{t\to\infty} \frac{x^{-2}}{-2}\text{from N to t} \\
                                             &= \lim_{t\to\infty} \frac{t^{-2}}{-2} - \frac{N^{-2}}{-2} \\
                                             &= \lim_{t\to\infty} -\frac{1}{2t^{2}} + \frac{1}{2N^{2}} \\
                                             &= \frac{1}{2N^{2}}
                                             \shortintertext{Now set this to be less than our goal} 
        \frac{1}{2N^{2}} &< 0.01
        \\
        1 &< 2N^{2} * 0.01 \\
        100 &< 2N^{2} \\
        50 &< N^{2}\\
        N^{2} &> 50 
\end{align*}
if N is greater than $\sqrt{50} $, then the sum is within 0.01 of the infinite sum
\fbox{\begin{minipage}{15em}
THIS ONLY WORKS IF IT SATISFIES THE INTEGRAL HYPOTHESIS!!!!
\end{minipage}}

\section*{Telescoping Series}
\[
        \sum_{n=4}^{\infty}(\frac{1}{n} - \frac{1}{n+2}) = \frac{1}{4} - \frac{1}{6} + \frac{1}{5} - \frac{1}{7} + \frac{1}{6}- \frac{1}{8} + \frac{1}{7} + \frac{1}{9}
.\] Everything cancels except for the first two first parts ($\frac{1}{4}$ and $\frac{1}{5}$ )

\subsection*{Web Assign-esque problem}

\begin{align*}
        \shortintertext{find value of} 
        \sum_{n=4}^{\infty} \frac{1}{n^{2}+n} 
        \shortintertext{How to guess: Partial Fraction Decomposition} 
        \frac{1}{n^{2} +n} &= \frac{1}{n(n+1)} \\
        \frac{1}{n(n+1 }&= \frac{A}{n} + \frac{B}{n+1}
        \shortintertext{reminder: if we multiply both sides by denominator $n(n+1)$ then things cancel such that}
        1 &= A(n+1) + B(n) \\
        \shortintertext{Easiest way: plug in "easy" values of n} 
        n &= 0\\
        1 = A*1 + B*0 \\
        A &= 1\\
        n &= -1 \\
        1 &= B(-1) \\
        B &= -1 \\
          &= \sum_{n=4}^{\infty} (\frac{A}{n} + \frac{B}{n+1}) \\
          &= \sum_{n=4}^{\infty} (\frac{1}{n} - \frac{1}{n+1})
\end{align*}


\section*{11.4 Final thing: Comparison Test}
If $0 \leq x_n \leq y_n$, then \\
if $\sum_{n=1}^{\infty}y_n$ converges, then $\sum_{n=1}^{\infty}x_n$ converges \\
if $\sum_{n=1}^{\infty}x_n$ diverges, then $\sum_{n=1}^{\infty}y_n$ diverges

\fbox{\begin{minipage}{35em}
                Key things we want to compare these two: (Because they're what we know)
                \begin{enumerate}
                        \item $\frac{1}{n^{p}}$ \[ \begin{cases}
                                                conv & p > 1
                                                div & p \leq 1
                        \end{cases}
                        \]
                        \item $r^{n}$
                \end{enumerate}
\end{minipage}}

\subsection*{EX}
\begin{align*}
        \sum_{n=4}^{\infty}\frac{lnn}{n} \\
        &=  \frac{lnn}{n} > \frac{1}{n}
        &= n \geq 4, ln n \geq ln 4 > 1 
        \shortintertext{so the original sum is also divergent by comparison test} 
\end{align*}
\end{document}
