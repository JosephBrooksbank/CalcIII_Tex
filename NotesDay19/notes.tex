\documentclass[]{report}

\usepackage{graphicx}
\usepackage{subcaption}
\usepackage{float}
\usepackage{amsmath}
\usepackage{mathtools}

% Title Page
\title{Calc notes Day 19}
\author{Joseph Brooksbank}

\begin{document}
\maketitle
\section*{11.1 Sequences}
Written as \textbf{($X_N$)}
\subsection*{Convergence}

\noindent\fbox{%
    \parbox{\textwidth}{%
    EX:}%
}
$ \sqrt[2]{n + 2} $
$
\begin{cases}
        x_1 & \sqrt[2]{3} \\
        x_2 & \sqrt[2]{4} \\
        x_3 & \sqrt[2]{5} \\  

\end{cases} etc...
$
Another way to describe sequence: recursive def:
\[
        x_{n=1} = \frac{1}{1 + x_n}, x_1 = 1
.\] 
\\EX:\\
\[
     x_2  = \frac{1}{1+x_1}= \frac{1}{1+1}=\frac{1}{2}
.\]  

\noindent\fbox{%
    \parbox{\textwidth}{%
    what does it mean to say seq $x_n$ approaches / converges to a limit ?}%
}
\\
\noindent\fbox{%
    \parbox{\textwidth}{%
    \textbf{IDEA} $(X_n) \to L $ means:}%
} No matter how close you want to get to $x_n$ to get to L, it will happen if n is large enough
\\
EX : $(\frac{1}{n})$ 
\\
$1, \frac{1}{2}, \frac{1}{3}, \frac{1}{4}. \frac{1}{5}$
\\ This converges to 0.
\\
Reason: for any tiny number  you give me, if I go far enough in $(x_n)$, then $x_n$ is that close to 0. 
\subsubsection*{Illustration}
Can we make $(\frac{1}{n})$ be within 0.1 of 0?
\\ 
if  $n > 10 \to 0 < \frac{1}{n} < \frac{1}{10} $ 
\\ 
this means that dist from $\frac{1}{n}$ to 0 is $< \frac{1}{10}$ after n = 10. 
\\ This tells us that we can get beneath 0.1, but so what? 
\\
\textbf{EX part 2:}
\\
What about under 0.0005?
\\
\clearpage
\subsubsection*{Convergence in "math speak"}
For every $\epsilon > 0$, then there is N such that if $n > N$, then the distance between $x_n$ and L is $< \epsilon$
if $n > 2000$, then $0 < \frac{1}{n} < 0.0005$
\\
\begin{itemize}
        \item If you go FAR ENOUGH, (past N), then n will be less than any given bounds 
\end{itemize}
\textbf{This is a lot to write, how can we show that $(x_n) \to L$ without doing that?}
2 Techniques to show $(x_n)\to L$ :
\begin{itemize}
        \item Algebra + simplification + $\frac{1}{n} \to 0$

\end{itemize}
                \[
                        \frac{2^{2} + n }{2n^{2}- 1} = \frac{(n^{2} + n) * \frac{1}{n^{2}}}{(2n^{2}-1)*\frac{1}{n^{2}}} = \frac{1 + \frac{1}{n}}{2 - \frac{1}{n} * \frac{1}{2}} = \frac{1 + 0}{2 - 0 * 0} = \frac{1}{2} 
                .\] 
                \begin{itemize}
                        \item Calc / L'H
                                \subitem \textbf{EX:} What's the limit of  $\frac{ln(n)}{n}= x_n$?
\end{itemize}

\noindent\fbox{%
    \parbox{\textwidth}{%
    ONLY POSITIVE INTEGERS ARE ALLOWED FOR n in SEQUENCES }%
}
Moving back to the problem above: 
\\
If we graph $\frac{ln(n)}{n}$ : We can only plug in positive \textbf{INTEGERS}. See notebook CALC III day 18 and 19 for fig 1.
\\
The graph of $\frac{ln(n)}{n}$ is dots, which are a small part of graph $\frac{ln(x)}{x}$. 
\\
Can treat this like a \textbf{FUNCTION} anyway, and can use calculus on it:
\[
        \lim_{x\to\infty} \frac{ln(x)}{x} \to \frac{\infty}{\infty} \to \lim_{x\to\infty} \frac{\frac{1}{x}}{1} = 0
.\] 
Therefore: 
\\
\[
        \lim_{n\to\infty} \frac{ln(n)}{n} \to 0
.\] too!

\noindent\fbox{%
    \parbox{\textwidth}{%
    \textbf{RULE:} If you can write the terms of $x_n$ as values of a function $f(n), \lim_{x\to\infty} f(x) = L$ then $(x_n) \to L$}%
}
What can't be turned into a function? 
\\
\noindent\fbox{%
    \parbox{\textwidth}{%
    n! is an example of a sequence with CANNOT be turned into a function f(x) (in a simple way)}%
}
        
\subsection*{Increasing $/$ Decreasing}
\noindent\fbox{%
    \parbox{\textwidth}{%
    What does it mean to say $(x_n)$ is increasing?}%
}
For every n, $x_{n+1} > x_n$
\\
Decreasing? 
\\
For every n, $x_{n+1} < x_n$
\noindent\fbox{%
    \parbox{\textwidth}{%
    How to decide if seq is increasing or decreasing? }%
}    $x_n = \frac{n}{n+1}$ 
\\
rewrite as a function: 
\\
$\frac{x}{x+1} $ is a function, take derivative, derivative is inc 
\end{document}




