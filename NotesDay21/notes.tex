\documentclass[]{report}

\usepackage{graphicx}
\usepackage{subcaption}
\usepackage{float}
\usepackage{amsmath}
\usepackage{mathtools}

% Title Page
\title{Calc III Notes Day 21}
\author{Joseph Brooksbank}

\begin{document}
\maketitle
\section*{Sequences}
\[
        (x_n) = x_1,x_2,x_3...
.\]
\subsection*{How to check if $(x_n)$ converges to limit as $(n \to  \infty)$?}
\begin{list}{}
        \item Simplification
                \noindent\fbox{%
                    \parbox{\textwidth}{%
                    EX 1:}%
                }
                \[
                        x_n = \frac{n^{2}}{ \sqrt[2]{n^{6}+ 1} }
                .\] 
                Next: put largest n power on the bottom
                \begin{align}
                        \frac{n^{2}}{ \sqrt[2]{n^{6}+1} } * \frac{\frac{1}{n^{3}}}{\frac{1}{n^{3}}} \\
                        &= \frac{\frac{1}{n}}{ \sqrt[2]{\frac{n^{6}+1}{n^{6}}} }
                        \shortintertext{$\frac{1}{n} \to$ 0}
                        &= \frac{0}{ \sqrt[2]{1 + \frac{1}{n^{6}}} } 
                        \\
                        &= \frac{0}{ \sqrt[2]{1 + 0} }
\\
                        &= \frac{0}{1}
                        \\
                        &= 0 
                \end{align}
     \item L'H Rule 
             \\
             We already did this so 
            \item Squeeze Theorem 
                    \\
                    Suppose we had 3 sequences, $x_n \leq y_n \leq z_n$ 
                    \\
                    and 
                    \\
                    $x_n \to L$ 
                    \\
                    $z_n \to L$ 
                    \\
                    then $y_n \to L$ 

                    \noindent\fbox{%
                        \parbox{\textwidth}{%
                        Example of Squeeze Theorem}%
                    } 
                    \[
                            y_n = \frac{sin(n)}{n}
                    .\] 
                    This equation is totally unpredictable (What is sin(11)?)... but \textbf{BETWEEN} -1 and 1 
                    \\
                    \large{IDEA: squeeze the equation between other things}
                    \[
                            -1 \leq \frac{sin(n)}{n} \leq 1
                    .\] 
                    The squeeze theorem needs to have both sides of the squeeze go to the same point 
                    \\
                    so instead:
                    \[
                            -\frac{1}{n} \leq \frac{sin(n)}{n} \leq \frac{1}{n}                    
                    .\] 
                    Both $-\frac{1}{n} and \frac{1}{n}$ $\to$ 0, so by squeeze theorem, so does $\frac{sin(n)}{n}$ !
\end{list}
\subsection*{How to check if $(x_n) inc / dec$?}
\begin{enumerate}
        \item Compare $x_n$ and $x_{n+1}$ 
                \\
                EX: Is $x_n = \frac{n}{n+1}$ inc, dec, or neither?
                \\
                \begin{itemize}
                        \item $x_1 $ = $\frac{1}{2}$ 
                                \item $x_2 = \frac{2}{3}$ 
                                        \item $x_3 = \frac{3}{4}$
                \end{itemize}
                Guess: Increasing!
                \\
                $x_n < x_{n+1}$
                \\
                $\frac{n}{n+1} < \frac{n+1}{n+2}$
                Cross Multiply:
                \\ $n(n+2) < (n+1)(n+1)$
                \\ Check to make sure that this is true, which it is 
        \item Check if derivative is increasing or decreasing 
                \\
                Went over this last time so we're not doing it this time, but its pretty self explanatory 
        
\end{enumerate}

\subsection*{$(x_n)$ bounded from above $/$ below?}
          \noindent\fbox{%
              \parbox{\textwidth}{%
              We say $x_n$ is bounded from above IF:}%
          } \textbf{There's a number M so that $x_n \leq$ M for all n}
          \\
          Same idea for below, but..from...below 

          \subsubsection*{How to check if $x_n$ is bounded from above / below? } 
          \begin{enumerate}
                  \item "common sense" / functions known to be bounded 
                          \begin{itemize}
                                  \item \textbf{EX 1}
                                          \\
                                          \[
                                                  x_n = sin(n)
                                          .\] Bound by -1 and 1, so \textbf{BOUND ABOVE AND BELOW} 

                                  \item \textbf{EX 2} 
                                          \\
                                          \[
                                                  x_n = 4 cos(n+5) + 3
                                          .\] Also bound from both sides 

                                          \item \textbf{EX 3}
                                          \\
                                          \[
                                          x_n = n^{2}
                                          .\] Bound from below 
                                          \\ x is always bigger than 0, so its bounded on the bottom but not the top 
                          \end{itemize}
                          \item If sequence converges, then its bounded from both sides! 
                          
          \end{enumerate}
          \noindent\fbox{%
              \parbox{\textwidth}{%
              \textbf{UNBOUND THINGS:} \\
      $x_n = n * (-1)^{n}$ Goes to both -infinity and positive infinity }%
          }

\end{document}
