\documentclass[12pt]{article}

\usepackage{import}
\usepackage{xifthen}
\usepackage{pdfpages}
\usepackage{transparent}

\usepackage{geometry}
\usepackage{graphicx}
\usepackage{subcaption}
\usepackage{float}
\usepackage{amsmath}
\usepackage{mathtools}
\newcommand{\incfig}[1]{%
    \def\svgwidth{\columnwidth}
    \import{./figures/}{#1.pdf_tex}
}

% Title Page
\title{Calc III Notes Day 32}
\author{Joseph Brooksbank}

\begin{document}
\maketitle

\fbox{\begin{minipage}{18em}
Exam Friday 
Material after exam 2 up to 11.5 (ALT Series Test)
\end{minipage}}

\section*{One last thing about AST}
\begin{align*}
        1 - 0 + \frac{1}{2} - 0 + \frac{1}{3} - 0 + \frac{1}{4}
        \shortintertext{This is alternating} 
        \shortintertext{The terms $\to$ 0} 
        \shortintertext{So it should converge, but it actually diverges} 
        \shortintertext{basically just $\sum_{}^{} \frac{1}{n} $} 
        \shortintertext{It shoudn't actually converge, $x_n$ is NOT decreasing} 
\end{align*}

\begin{align*}
\fbox{\begin{minipage}{15em}
If the "$x_n$ " decreasing hypothesis fails, then alt series might diverge!
\end{minipage}}
\end{align*}
\clearpage
\section*{11.6 Absolute Convergence, Ratio Test, and other things}
\begin{align*}
        \text{Look at series} \\
        1 - \frac{1}{2} + \frac{1}{2} - \frac{1}{3} + \frac{1}{3} - \frac{1}{4} + \frac{1}{4} - \frac{1}{5}...
        \shortintertext{This series converges by AST, but also by telescoping (everything cancels except 1} 
        \text{"Positive" part of series: $1 + \frac{1}{2} + \frac{1}{3} + \frac{1}{4}...$}\\
        \text{"negative part": $-\frac{1}{2} -\frac{1}{3} -\frac{1}{4}$} 
        \shortintertext{Diverges, going to negative $\infty$} 
\end{align*}
This series "barely converges" because the pos $\infty$ and neg $\infty$ barely "canel out"

KIND of like an indeterminate form 

\subsubsection*{Contrasting with}
\begin{align*}
        1 - \frac{1}{2^{2}} + \frac{1}{2^{2}} - \frac{1}{3^{2}} + \frac{1}{3^{2}} - ...
        \shortintertext{Both positive and negative parts converge seperately, $\sum_{}^{} \frac{1}{n^{2}} $ converges} 
\end{align*}

\fbox{\begin{minipage}{40em}
                DEFINITION: \\
                A series $\sum_{n=1}^{\infty} x_n $ \textbf{absolutely} converges if
                \begin{itemize}
                        \item pos part and neg part both converge  
                \end{itemize}
                or can be written as $\sum_{n=1}^{\infty} |x_n| $ conv $<---$ making both parts pos, and still conv
\end{minipage}}

\subsubsection*{Ex}
\begin{align*}
        \sum_{n=1}^{\infty} \frac{(-1)^{n}}{\sqrt{n} }
        \\ \text{Does this converge?} 
        \\ \text{AST...?} \frac{1}{\sqrt{n} } 
        \\ \text{going to 0? yes}
        \\ \text{decreasing? yes, $\sqrt{n} $ is increasing, so decreasing } 
        \shortintertext{So this entire thing does converge} 
        \shortintertext{Does it absolutely converge?} 
        \sum_{n=1}^{\infty} | \frac{(-1)^{n}}{\sqrt{n} }| \\
        = \sum_{n=1}^{\infty} \frac{1}{\sqrt{n} } 
        \shortintertext{p is not bigger than 1, so it diverges} 
\end{align*}
\fbox{\begin{minipage}{25em}
                DEFINITION: \\
                \textbf{Conditionally} Convergent means:
                \begin{itemize}
                        \item convergent but NOT absolutely convergent  
                \end{itemize}
\end{minipage}}

\subsubsection*{EX}
\begin{align*}
        \sum_{n=1}^{\infty} \frac{cos(n)}{n^{2}} \text{  Converges?}  
\end{align*}
\begin{itemize}
        \item cos 1 = + 
        \item cos 2 = -
        \item cos 3 = -
        \item cos 4 = -
        \item cos 5 = +
\end{itemize}
Not positive all the time, and also doesn't alternate, sign oscillates "randomly" 
\\
Can't use integral test, alternating test, etc 
\\
Let's try... absolute convergence? 
\begin{align*}
        \sum_{n=1}^{\infty} | \frac{cosn}{n^{2}}| &= \sum_{n=1}^{\infty} \frac{|cos n|}{n^{2}} \text{because $n^{2}$ is always pos}   
        \\
        \frac{|cosn|}{n^{2}} &\leq \frac{1}{n^{2}}\\
        \sum_{n=1}^{\infty} \frac{1}{n^{2}} converges
        \shortintertext{ due to the comparison test, the ABS series converges (always postive) converges due to comparison test with $\frac{1}{n^{2}}$.} 
\end{align*}
\fbox{\begin{minipage}{15em}
Absolute Convergence is even \textbf{STRONGER} than convergence!
\end{minipage}}
So original series converges, because absolute converges! cool! 

\subsection*{This chapter is really about}
\section*{RATIO / ROOT TESTS}
\begin{align*}
        \sum_{n=1}^{\infty} \frac{n^{2}}{3^{n}} \text{  Does this converge or diverge?}   
\end{align*} Which wins? $3^{n}$. 
Exponential growth CRUSHES polynomial growth \\
\fbox{\begin{minipage}{15em}
The $3^{n}$ should make $n^{3}$ irrelevant, so that makes this "like a geometric series"
\end{minipage}}
\\
\fbox{\begin{minipage}{15em}
                Idea: How to tell if $\sum_{n=1}^{\infty} x_n $ is "basically geometric"?
\end{minipage}}
\\
If this is geometric, then $x_n$ is basically $r^{n}$ $\to$ $\sqrt[n]{x_n} == r$
\\
\fbox{\begin{minipage}{25em}
ROOT TEST:\\
If $\sum_{n=1}^{\infty} x_n $ series and $\sqrt[n]{x_n} \to L$, \\
Then 
\begin{enumerate}
        \item if $ L < 1$ originial series (absolutely) converges
        \item if $1 < L$, then series diverges
        \item if $L = 1$ then test was inconclusive and you have to use another test
\end{enumerate}
\end{minipage}}


\end{document}
