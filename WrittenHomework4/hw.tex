\documentclass[]{article}
\usepackage{graphicx}
\usepackage{subcaption}
\usepackage{float}
\usepackage{amsmath}
\usepackage{mathtools}
\usepackage{fullpage}

% Title Page
\title{Math 1953 Written Homework 4}
\author{Joseph Brooksbank}

\begin{document}
\maketitle


\noindent\fbox{%
    \parbox{\textwidth}{%
    HELPFUL\\
    $\int_{0}^{1} \frac{1}{x^{p}} dx $ Converges if $p < 1$\\
    $\int_{1}^{\infty} \frac{1}{x^{p}} dx $ Converges if $p > 1$}%
}
\begin{enumerate}
        \item For which values of $p$ does the improper integral $\int_{e}^{\infty} \frac{1}{x(ln(x))^{p}} dx $ converge, and what is its value (in terms of $p$)?
                                \begin{align*}
                                        \int_{e}^{\infty} \frac{1}{x(lnx)^p} dx 
                                        &= \lim_{t\to\infty} \int_{e}^{t} \frac{1}{x(lnx)^{p}} dx \\
                                        \shortintertext{Assuming $\lim_{t\to\infty}  $ for next steps} 
                                        &= \int_{e}^{t} \frac{(ln(x))^{-p}}{x} dx \\
                                        \shortintertext{Using u substitution, with $u = ln(x)$ and $du = \frac{1}{x}dx$}
                                        &= \int_{1}^{ln(t)} (u)^{-p} du 
                                        \\
                                        &= \frac{u^{1-p}}{1- p} \text{from ln(x) to 1} \\
                                        &= \frac{ln(t)^{1-p}}{1-p }- \frac{1^{1-p}}{1-p}
                                        \shortintertext{If $p > 1$, then $ln(t)^{negative} \to 0$}
                                        \\
                                        \shortintertext{Thus, it converges to $-\frac{1}{1-p}$} 
                                \end{align*}
\clearpage
        \item   $\int_{0}^{\infty} \frac{1}{x^{3} + \sqrt[2]{x} } dx $
                \begin{align*}
                        &= \int_{0}^{1} \frac{1}{x^{3} + \sqrt[2]{x} } dx + \int_{1}^{\infty} \frac{1}{x^{3}+\sqrt[2]{x} } dx \\
                        \shortintertext{Prove that both of these can be bounded under something:} 
                        &= \frac{1}{x^{3}+\sqrt[2]{x} } \leq \frac{1}{x^{3}}\\
                        \shortintertext{Because of the information in the box written at the top of the page, $\int_{1}^{\infty} \frac{1}{x^{3}+\sqrt[2]{x} } dx $ converges} 
                        \shortintertext{Also} 
                        &= \frac{1}{x^{3}+\sqrt[2]{x} } \leq \frac{1}{x^{\frac{1}{2}}}
                        \shortintertext{and, just like the first part, the information in the box tells us that this means $\int_{0}^{1} \frac{1}{x^{3}+\sqrt[2]{x} } dx $ converges.} 
                \end{align*}
                Because both parts of the integral can be compared against upper bounds which converge, due to the comparison test they also converge. And since both parts of the integral converge, the entire integral does. 

     \item 
             (this is going to be fun to write in LaTeX)
             \begin{enumerate}
                     \item 
                             \[ \begin{cases}
                                             1 & 1\\
                                             2 & 1 + \frac{1}{1} \\
                                             3 & 1 + \frac{1}{1+\frac{1}{1}} \\
                                             4 & 1 + \frac{1}{1+\frac{1}{1+\frac{1}{1}}} \\
                                             5 & 1 + \frac{1}{1+\frac{1}{1+\frac{1}{1+\frac{1}{1}}}} \\
                                     \end{cases}
                             \]
                     \item   
                             \[ \begin{cases}
                                             1 & 1\\
                                             2 & 2\\
                                             3 &  \frac{3}{2}\\
                                             4 &  \frac{5}{3}\\
                                             5 & \frac{8}{5}
                             \end{cases}
                             \]
                             There is a pattern, $x = 1 + \frac{1}{x_{-1}}$

                     \item   $1, 2, 1.5, 1.6, 1.6, 1.625, 1.615,,,,$ appears to be approaching 1.61ish 
                             \\
                             After googling, it is approaching \textbf{1.61803398875}, also known as the \large{\textbf{Golden Ratio}}
                             
             \end{enumerate}
\end{enumerate}

\end{document}
