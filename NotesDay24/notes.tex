\documentclass[]{article}

\usepackage{geometry}
\usepackage{graphicx}
\usepackage{subcaption}
\usepackage{float}
\usepackage{amsmath}
\usepackage{mathtools}

% Title Page
\title{Calc III Notes Day 24}
\author{Joseph Brooksbank}

\begin{document}
\maketitle
\section*{Main topics for test}
\begin{itemize}
        \item Improper Integrals   
     \item Sequences
             \subitem Convergent?
             \subitem inc dec neither?
             \subitem Bounded above below both neither?

             \item Series
                     \subitem Def of a series 
                     \subitem geometric 
                     \subitem is series converge, then terms $\to$ 0 
\end{itemize}

\section*{Reivew for test}
\subsection*{ 3.c from practice exam}
\[
x_n = \frac{n}{n^{2} + 1}
.\]

\begin{itemize}
        \item Does x n converge? \\
                $(x_n)$ can be "turned into a function of x" $\frac{x}{x^{2}+1}$ \\
                \begin{align*}
                        \lim_{x\to\infty} \frac{x}{x^{2} + 1} \\
                        &= \frac{\infty}{\infty} \\
                        \shortintertext{L'H} 
                        &= \frac{1}{2x} \\
                        &= \lim_{x\to\infty} \frac{1}{2x} = 0  
                \end{align*}
                $(x_n)$ converges to zero 
        \item is xn inc / dec?  
                \\
                Take derivative 
                \begin{align*}
                        \frac{x}{x^{2} + 1} \\
                        &= (x)^{'}(x^{2}+1)-x(x^{2}+1)^{'} / (x^{2} + 1)^{2} 
                        \\
                        &= \frac{1-x^{2}}{\text{Some positive number, its squared} } \\
                        \shortintertext{We're really looking at a sequence, so x is always bigger than one} 
                        &= 1 - x^{2} \leq 0 
                        \\
                        &= Derivative is negaitve, so function is dec, so sequence is decreasing 
                \end{align*}
        \item is xn bounded above / below ?  \\ 
                Seq is convergent, so the sequence is bounded 
\end{itemize}
\subsection*{Practice Exam Question 4}
\[
\sum^{\infty}_{n = 0} \frac{3^{n+1}}{5^{n-1}}
.\] 

plug in n = 1 
\begin{align*}
        &= \frac{3^{2}}{5^{0}}
        \shortintertext{Next term} 
        &= \frac{3^{3}}{5^{1}}
        \shortintertext{next term} 
        &= \frac{3^{4}}{5^{2}} \\
        &= \frac{3^{2}}{5^{0}}(1 + \frac{3}{5} + \frac{3^{2}}{5^{2}}+...)\\
        \shortintertext{This turns into our formula with r between -1 and 1 } 
        &= \frac{3^{2}}{5^{0}}* (\frac{1}{1 - \frac{3}{5}})
\end{align*}
\subsection*{WebAssign 3?}
\begin{align*}
        \lim_{x\to-\infty} x^{2} * e^{2x} \\
        &= \infty * 0 
        \shortintertext{this is an indeterminate form} 
        &= \frac{x^{2}}{e^{-2x}} \\
        &= \frac{2x}{-2e^{-2x}} \\
        &= \frac{2}{4e^{-2x}} \\
        &= \frac{2}{\infty} = 0 
\end{align*}
\subsection*{WebAssign 3 q 15}
\[
\int_{0}^{3} \frac{18}{x^{2} - 6x + 5} dx 
.\] 
Can plug in both end points, and its fine 
\\
However, there ARE bad points 
\\
DNE if $x^{2} - 6x + 5 = 0$
\\
(x -5)(x -1) = 0 
\\
5 is not in the integral but 1 is 
\begin{align*}
        \int_{0}^{1} dx + \int_{1}^{3}  dx 
        \shortintertext{this is an ugly one so} 
\end{align*}
\subsection*{Comparison TEST}
\[
\int_{1}^{\infty} \frac{2x^{2}}{x^{5}+10} dx 
.\] 
    The only improper integral we have been taught are 
    $\int_{1}^{\infty} \frac{1}{x^{p}} dx $
    \[ \begin{cases}
                    conv & p > 1 \\
                    diverge & p\leq 1 
    \end{cases}
    \]
\end{document}
