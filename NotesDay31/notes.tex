\documentclass[12pt]{article}

\usepackage{import}
\usepackage{xifthen}
\usepackage{pdfpages}
\usepackage{transparent}

\usepackage{geometry}
\usepackage{graphicx}
\usepackage{subcaption}
\usepackage{float}
\usepackage{amsmath}
\usepackage{mathtools}
\usepackage{fontspec}
\setmainfont{Iosevka SS01}

\newcommand{\incfig}[1]{%
    \def\svgwidth{\columnwidth}
    \import{./figures/}{#1.pdf_tex}
}

% Title Page
\title{CALC III Day 31}
\author{Joseph Brooksbank}

\begin{document}
\maketitle

\section*{Alternate Series Test}

\fbox{\begin{minipage}{35em}
                if series $\sum_{n=1}^{\infty} (-1)^{n}x_n $ or $\sum_{n=1}^{\infty} (-1)n+1 x_n$ and $x_n \geq 0$, series is \textbf{alternating.}

                If:
                \begin{enumerate}
                \item $x_n$ $\to$ 0
                \item $x_n$ is decreasing  
                \end{enumerate}
                then series is converging!

                if 1 fails, series diverges \\ if 2 fails, ???
\end{minipage}}

\subsubsection*{EX}

1:
\begin{align*}
        \sum_{n=2}^{\infty} \frac{1}{ln n} * (-1)^{n}\\
        &= \lim_{x\to\infty} \frac{1}{lnx} = \frac{1}{\infty} = 0 
\end{align*}
\clearpage
2:Decreasing?

\begin{align*}
        \frac{1}{lnx} &= \frac{0 * lnx - 1 * \frac{1}{x}}{(lnx)^{2}} 
        \\
                      &= \frac{\frac{-1}{x}}{(lnx)^{2}} 
                      \shortintertext{-1 is negative, so we need to make sure rest is positive} 
                      \shortintertext{x is always positive, and ln x is squared so its always positive so we good } 
\end{align*}

Other way to tell if its decreasing:

\begin{figure}[ht]
    \centering
    \incfig{graph-of-ln-x}
    \caption{graph of ln x}
    \label{fig:graph-of-ln-x}
\end{figure}

ln x is increasing, so $\frac{1}{increasing}$ is decreasing 

\fbox{\begin{minipage}{15em}
                Going to 0 and decreasing, so by AST its converging :) 
\end{minipage}}

\subsubsection*{EX}

\begin{align*}
        \sum_{n=1}^{\infty} \frac{(-1)^{n}}{n^{2}}&
        \shortintertext{converges to a number due to the AST} 
        \shortintertext{could we approximate to within 0.05?} 
        \shortintertext{the only way to get a guess / estimate for infinite series is to add up a bunch of terms} 
        \shortintertext{How many terms N do we need to add up to be certain that we're within $\frac{1}{N^{2}}$ of true value?} 
        \shortintertext{if add up the firt N turns, how far off am I from th eactual infinite series?} 
        \shortintertext{lets try 15 terms, how close am I to the infinite series?}
        \frac{1}{1^{2}} - \frac{1}{2^{2}} + \frac{1}{3^{2}} - .... + \frac{1}{15^{2}}     
        \shortintertext{how far apart is this from the infinite series? Take the series} 
        (\frac{1}{1^{2}} - \frac{1}{2^{2}} + \frac{1}{3^{2}} - .... + \frac{1}{15^{2}}) - (\frac{1}{1^{2}} - \frac{1}{2^{2}}..)
        \shortintertext{everything from the first term cancels, so we're left with} 
-         -\frac{1}{16^{2}} + \frac{1}{17^{2}} -... \\
                                                  = \frac{1}{16^{2}} - \frac{1}{17^{2}} + \frac{1}{18^{2}}...
                                                  \shortintertext{this is less than $\frac{1}{16^{2}}$} 
                                                  \shortintertext{Remember from yesterday, when we pair things together} 
                                                  \shortintertext{less than $\frac{1}{16^{2}}$, and 0.05 is much larger than $\frac{1}{16^{2}}$ so we good } 
                                                  \shortintertext{what we wanted: a number that has distance of less than 0.05 from value of infinite series} 
\end{align*}
\fbox{\begin{minipage}{15em}
What did we learn from all of this? 
\end{minipage}}

\\
\begin{align*}
        \shortintertext{If we have a series }
        \sum_{n=1}^{\infty} (-1)^{n} x_n \text{and 1 and 2 from AST are true} \\
        \shortintertext{then using 1st N terms gives estimate within distance $x_{N+1}$ of from value of infinite series} 
\end{align*}
So if you're trying to estimate the infinite series to within some number:
\begin{enumerate}
        \item Figure out how much N must be to make $X_{N+!} <$ that number  
        \item then our guess is the sum of the first N terms 
\end{enumerate}

\subsubsection*{EX}
\[
        \text{Approx} \sum_{n=2}^{\infty} \frac{(-1)^{n}}{ln n} \text{ to within 0.01}  
.\] 

\end{document}

