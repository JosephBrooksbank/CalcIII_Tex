\documentclass[12pt]{article}

\usepackage{geometry}
\usepackage{graphicx}
\usepackage{subcaption}
\usepackage{float}
\usepackage{amsmath}
\usepackage{mathtools}
\usepackage{fontspec}
\setmainfont{Iosevka SS01}

% Title Page
\title{Calc III Notes Day 26}
\author{Joseph Brooksbank}

\begin{document}
\maketitle

Notes: Web Assign 5 involves section 11.3,\\
postponed until Friday. 
\\
Questions 6-8 not doable yet!
\\
Inf Series
Questions:
\begin{list}{}
\item Does it converge (add up to some finite number?) 
\item or Diverge (doesn't add to some finite number)?  
\end{list}
So far, we know 
\begin{enumerate}
\item How to check if geom. series converges / diverges 
\item  $(x_n)$ $\to$ 0, then $ \sum^{n}_{i=1} x_n$ conv?
        \subitem no, if $\sum^{n}_{i=1} x_n$ conv, then terms $x_n$ $\to$ 0 
        \subsubitem I.E, if $x_n$ does not $\to$ 0, then $ \sum^{n}_{i=1} x_n$ diverges
        \subsubitem The alternative is not neccesarily true
        \subsubitem EX: $1 + \frac{1}{2} + \frac{1}{2} + \frac{1}{4} + \frac{1}{4} + \frac{1}{4} + \frac{1}{4} +....$ 
        \subsubitem The terms go to zero, but the \textbf{sum} still diverges ( 1 / 2 + 1 /2 = 1, etc) 
\end{enumerate}
\noindent\fbox{%
    \parbox{\textwidth}{%
    If the terms go to 0, the serives might converge OR it might diverge, rest of quarter is how to check this }%
}
Big picture: finding out if the denominator or numerator goes to infinity faster 
\\
Stepping back for a bit..
\\
\[
        \sum^{\infty}_{n=1} \frac{1}{n} = \frac{1}{1} + \frac{1}{2} + \frac{1}{3} + \frac{1}{4} +....
.\] 
This sum is the area of all rectangles in this infinite sum 
\\
Area of each rectangle is greater than the area under the curve y = 1 / x 
\\
Doing the integral of y = 1 / x gives a diverging improper integral, which by the comparison test means that the thing larger than it also diverges (goes to infinity) 
\\
EX 2:
\[
\sum_{n=1}^{\infty} \frac{1}{n^{3}} = \frac{1}{1^{3}} + \frac{1}{2^{3}} + 1^{3^{3}} 
.\] 
If we set the rectangles to start area from the left, \\
These rectangles all fit under $y = \frac{1}{x^{3}}$ 
\\
IE purple area $>$ sum of series 
\\
However if area starts to the left, the integral is $\int_{0}^{\infty} \frac{1}{x^{3}} dx $ Which fails 
\\
Instead, we can "chop off" the first term which equals 1 
\\
and do $\int_{1}^{\infty} \frac{1}{x^{3}} dx $ which equals $\frac{1}{2}$.
\\ 
Since all the boxes are less than the integral, the sum from 2 to $\infty$ is less than 1 / 2. 
\\
adding back the 1 from the first term, then the entire thing is $< \frac{3}{2}$. 
\\
Thus, the original series converges.
\section*{Putting all of this together: Integral Test}
How do decide whether something converges or diverges with integrals?\\

Say $\sum_{n=1}^{\infty} x_n $ satisfies:
\begin{enumerate}
        \item We can turn func into f(x). 
        \item $x_n$ decreasing 
        \item $(x_n)$ is positive  
\end{enumerate}
Then convergence status of $\sum_{n=1}^{\infty} x_n $ is SAME as the convergence status of $\int_{1}^{\infty} f(x) dx $. 
\\
aka
\\
\[ \begin{cases}
                \text{integral converges} & \text{series converges} \\
                \text{integral diverges}  & \text{series diverges} 
\end{cases}
\]

\subsection*{Example:}
does series \[
        \sum_{n=3}^{\infty} \frac{ln(n)}{n}
.\] Converge or diverge?

\begin{enumerate}
\item turn $x_n$ into function: 
        \subitem $f(x) = \frac{ln x}{x}$ 
        \item $\frac{ln n}{n}$ decreasing?
                \subitem Usually take first derivative of f(x)  
                \item are terms $\frac{ln n}{n} > 0$? 
                        \subitem all n are greater than 3, which implies that ln n > 0, and n > 0, so everything is greater than 0  
\end{enumerate}

\end{document}
