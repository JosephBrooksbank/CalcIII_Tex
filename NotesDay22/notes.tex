\documentclass[]{article}

\usepackage{geometry}
\usepackage{graphicx}
\usepackage{subcaption}
\usepackage{float}
\usepackage{amsmath}
\usepackage{mathtools}

% Title Page
\title{Calc III Notes Day 22}
\author{Joseph Brooksbank}

\begin{document}
\maketitle
\section*{Homework help}
Question 6 HW 3: 
$\lim_{x\to_0}  \frac{cos}{sin} - \frac{1}{x} $

\section*{Last Time:}
Prof said "If $(x_n) \to L$, then $(x_n)$ is bounded from above AND below
\\
This makes sense, after some number of terms (say 5 as an example), all of the terms must be within some interval sufficiently close to L 
\\ 
When the sequence is at this point, the sequence is \textbf{DEFINITELY} bounded (it has to be within that interval)
\\
If we add in the terms before that point, its \textbf{STILL} bounded, just by larger bounds.. But because all points PAST a certain n are bounded, later terms will never be "farther" outside of bounds than the ones before n.
\\
\noindent\fbox{%
    \parbox{\textwidth}{%
    Basically, just know that if x(n) $\to$ L, then x(n) is bounded from above AND below}%
}


\section*{11.2: Infinite Series!}
\noindent\fbox{%
    \parbox{\textwidth}{%
    An infinite series is the sum of an infinite list of numbers}%
    }
    \subsection*{Starting here}
    \textbf{EX 1:} all zeros:
    \[
    0, 0, 0, 0,... = 0
    \] 
    \textbf{EX 2}
    \[
    1 + 1 + 1 + 1...= \infty
    .\] 
    \textbf{EX 3}
    \[
    \frac{1}{2} + \frac{1}{4} + \frac{1}{8} + \frac{1}{16} + ... = 1
    .\] We SAY this adds up to 1. See Xeno's Paradox for more information (taking steps that are halfway to the next point each time)  
    \\ 
    What does this mean for infinite series?
    None of the items in the series are approaching 1 by themselves, but the SUM is approaching a limit at 1.
    \\
    \noindent\fbox{%
        \parbox{\textwidth}{%
        For any infinite series $x_1 + x_2 + x_3 +...$ the {nth partial sum} $S_n = x_1 + x_2 +...x_n$ We say the infinite series converges if the seq of partial sums converges.}%
    }

    \subsection*{Part 2: Electric Boogaloo}
    \textbf{EX: Take infinite series}
    \\
    \[
    0 + 0 + 0 + 0... =
    .\] Adding up n zeros is 0, no matter how many 
    \begin{align}
            S_1 &= 0 \\
            S_2 &= 0+0 \\
            S_3 &= 0+0+0 
    \end{align}
    {The $S_n$ converges to 0, so we say the inf. series converges / adds up to 0 } 
     \subsubsection*{EX 3}
     \[
     1 + 1 + 1 + 1 +....
     .\] Don't care about the \textbf{TERMS}, but about the \textbf{SUMS}

     \begin{align*}
             S_1 &= 1 \\
             S_2 &= 1 + 1 \\
             S_3 &= 1 + 1 + 1 \\
             S_4 &= 4... 
     \end{align*}
     The sums ARE moving towards infinity. 

     \subsubsection*{EX 4: A more interesting one}
     \[
             \frac{1}{10} + (\frac{1}{10})^{2} + (\frac{1}{10})^{3}....
     .\] This is the start of \textbf{GEOMETRIC SERIES!!}
    \\ 
     \noindent\fbox{%
         \parbox{\textwidth}{%
         Geometric series def: Adding up powers of a single number r}%
     }
     \[
     1 + r + r^{2} + r^{3} + ...
     .\] 
     \begin{align*}
             S_1 &= 1 \\
             S_2 &= 1 + r \\
             S_3 &= 1 + r + r^{2}\\
             S_n &= 1 + r +... + r^{n-1}
     \end{align*}
     \textbf{Question: do the $S_n$ approach limit? Converge?}
     \\
     Try multiplying $S_n * (1 - r)$
   \begin{align*}
           (1 + r +...+r^{n-1})(1-r) &= 1(1 + r +...) - r(1 + r +...) \\
                                     &= 1 + r +... + r^{n-1} - r - r^{2} - ....r^{n} \\
                                     &= 1 - r^{n}
                                     \shortintertext{Multiplying partial sum by (1 - r) gives us a final equation of $1-r^{n)}$} 
                                     \shortintertext{So now we get: $S_n = \frac{1-r^{n}}{1-r}$} 
   \end{align*} 
       QUESTION: What happens to $r^{n}$ as $r \to \infty$ ? \\
       Yes if $ -1 < r < 1 $
       \\ 
       No if  $r > 1 or r \leq -1$
       \\
       Converges when $-1 < r < 1$
\end{document}
