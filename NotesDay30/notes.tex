\documentclass[12pt]{article}

\usepackage{import}
\usepackage{xifthen}
\usepackage{pdfpages}
\usepackage{transparent}

\usepackage{geometry}
\usepackage{graphicx}
\usepackage{subcaption}
\usepackage{float}
\usepackage{amsmath}
\usepackage{mathtools}
\usepackage{fontspec}
\setmainfont{Iosevka SS01}

\newcommand{\incfig}[1]{%
    \def\svgwidth{\columnwidth}
    \import{./figures/}{#1.pdf_tex}
}

% Title Page
\title{CALC III Notes Day 30}
\author{Joseph Brooksbank}

\begin{document}
\maketitle

\section*{Homework help}

\subsection*{Written Homework 3}
\[
\sum_{n=2}^{\infty} \frac{1}{n!}
.\] Show converges with comparison test 
\\
Idea: show that the above sum is less than or equal to something that converges \\
\begin{align*}
        \shortintertext{What is $\frac{1}{n!}$ less than?}\\
        \frac{1}{n!} &\leq \frac{1}{2} \\
        \shortintertext{n! is greater than or equal to 2} 
        \shortintertext{Unfortunately, that isn't gonna help (that sum is still diverging)} 
        \frac{1}{n!} &\leq \frac{1}{n} 
        \shortintertext{this STILL doesn't converge lmao} 
        \frac{1}{n!} &\leq \frac{1}{n^{2}}
        \shortintertext{This IS true, but how do we prove it?} 
        \shortintertext{prove n! greater than n sqared}
        n! &\geq n(n-1) \\
           &= n^{2} - n 
           \shortintertext{Limit comparison with $\sum_{n=1}^{\infty} \frac{1}{n^{2}} $} 
\end{align*}

\section*{11.5 Alternating Series}
Alternating Series: Terms swap from pos, neg, pos, neg, etc 

\subsubsection*{EX}
\begin{align*}
        1 - \frac{1}{2} + \frac{1}{3} - \frac{1}{4} + \frac{1}{5}  
        \shortintertext{Question: Do the alternating signs cause enough 'cancelling" to make this converge?} 
        0 \leq series &\leq 1
        \shortintertext{can "group" every two pairs and they're always greater than 0} 
        \shortintertext{If we group them in pairs IGNORING 1, every single thing inside those groups are positive, and 1 minus a bunch of positive things is less than 1} 
\shortintertext{What if we grouped things after a point?} 
1 - \frac{1}{2} + \frac{1}{3} - \frac{1}{4} + (\frac{1}{5} - \frac{1}{6}) + (\frac{1}{7} - \frac{1}{8})... > 1 - \frac{1}{2} + \frac{1}{3} + \frac{1}{4} 
\shortintertext{Grouping after a different number} 
1 - \frac{1}{2} + \frac{1}{3}-(\frac{1}{4} - \frac{1}{5}) - (\frac{1}{6} - \frac{1}{7}) ... < 1 - \frac{1}{2} + \frac{1}{3} 
\shortintertext{Trapped series between two numbers $\frac{1}{4}$ apart: if we keep going, it turns out you can do this forever until the series converges } 
\end{align*}
Lets talk about why this works 
\\
Alternating series should converge because of the reasoning above\\
say series is  \[
        \sum_{n=1}^{\infty} (-1)^{n} x_n     x_n \geq 0
.\] 
or \[
        \sum_{n=1}^{\infty} (-1)^{n+1} 
.\]  
need some things:
\begin{enumerate}
        \item $x_n \to 0$  
        \item $x_n$ is decreasing   
\end{enumerate}
Then series converges!!

Thats all you need for the Alternating Series Test (AST)
\\
If a series is alternating, just ceck if it goes to 0 and is decreasing 
Much easier than limit or integral or comparison test, because if it follows these guidelines then it converges and you're done 

\subsubsection*{EX}
Does series \[
        \sum_{n=2}^{\infty} (-1)^{n} * \frac{n^{2}}{n^{3}+1} 
.\] Converge or Diverge?
\begin{align*}
        x_n &= \frac{n^{2}}{n^{3}+1} \\
        \shortintertext{just checking 2 things: does it go to 0, and is it decreasing?} 
        \shortintertext{turn into a function, take derivative} 
        \frac{n^{2}}{n^{3}+1} &\to  \frac{x^{2}}{x^{3} +1}\\
                              &\to L'H \lim_{x\to\infty} \frac{2x}{3x^{2}}\\
                              &= \lim_{x\to\infty} \frac{2}{3x} = 0 
                              \shortintertext{now check if decreasing} 
        (\frac{x^{2}}{x^{3}+1})^{'} &= \frac{2x(x^{3}+1) - x^{2}(3x^{2})}{(x^{3}+1)^{2} }\\
                                    &= \frac{x(2-x^{3}}{(x^{3}+1)^{2}}
                                    \shortintertext{Less than 0, so original function is decreasing (plug in numbers from sum) } 
\end{align*}
Its decreasing and going to 0, so its converging 

\subsubsection*{EX}
$\sum_{n=2}^{\infty} (-1)^{n} * \frac{n}{2n+1} $ 
\begin{enumerate}
\item $x_n \to 0?$ 
        \begin{align*}
                \frac{2 * \frac{1}{n}}{(2n+1) * \frac{1}{n}} &= \frac{1}{2 + \frac{1}{n}}
                \\
                                                             &\to \frac{1}{2}
        \end{align*}
\end{enumerate}
Terms don't go to zero, so it doesn't converge (this is a rule of all series, not just the Alternating Series)

\end{document}
