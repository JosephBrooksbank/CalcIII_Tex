\documentclass[12pt]{article}

\usepackage{import}
\usepackage{xifthen}
\usepackage{pdfpages}
\usepackage{transparent}

\usepackage{geometry}
\usepackage{graphicx}
\usepackage{subcaption}
\usepackage{float}
\usepackage{amsmath}
\usepackage{mathtools}
\usepackage{fontspec}
\setmainfont{Iosevka SS01}

\newcommand{\incfig}[1]{%
    \def\svgwidth{\columnwidth}
    \import{./figures/}{#1.pdf_tex}
}

% Title Page
\title{Math 1953 \\ Written Homework 7}
\author{Joseph Brooksbank}

\begin{document}
\maketitle

\begin{enumerate}
        \item Infinite Series $\sum_{n=1}^{\infty} (-1)^{n+1}\frac{1}{n}$ 
                \subitem Explain why alternating haromic series converges to a limit: \\
                    This converges fairly easily using the Alternating Series Test:
                    \begin{align*}
                            \shortintertext{1: $\to$ 0?} 
                            \text{this acts like $\frac{1}{n}$ when looking at if it goes to 0, and $\frac{1}{n} \to 0$}  
                            \shortintertext{Decreasing?} 
                            \text{$\frac{1}{n} > \frac{1}{n+1}$, so yes} 
                    \end{align*}

    \item Series within 0.1 of true value \\
            n=10 = exactly -0.1, so using n=11 for the N+1 in the formula $\sum_{n=1}^{N} x_n$ = within N+1 of true value
            \\
            $\sum_{n=1}^{11} (-1)^{n+1}\frac{1}{n}$ = 0.7365
    \item ln(2) = 0.6931, ln(2) + 0.1 = 0.7931, ln(2) < 0.7365 < ln(2) + 0.1
    \item Positive and Negative Part
            \subitem Pos part: \\
            \begin{align*}
                    1 + \frac{1}{3} + \frac{1}{5} + ... &= \sum_{n=1}^{\infty} \frac{1}{2n -1} \\
                    \text{LCT with $\frac{1}{n}$} &= \frac{n}{2n-1}. \text{limit using L'H } = \frac{1}{2}
                    \shortintertext{$\frac{1}{n}$ diverges, so so does the pos part} 
                    \\
            \end{align*} 
            \subitem Neg part: \\
            \begin{align*}
                    -\frac{1}{2} - \frac{1}{4} - \frac{1}{6} &= \sum_{n=1}^{\infty} -\frac{1}{2n} \\
                    \shortintertext{Similar LCT with $\frac{1}{n}$ gives us that $\frac{1}{2n}$ converges similarly to $\frac{1}{n}$, so the neg part also diverges}
            \end{align*}
\end{enumerate}

\section*{3 Ratio Test}
\begin{align*}
        &\sum_{n=1}^{\infty} \frac{(n!)^{2}}{(2n)!}
        &=\frac{\frac{((n+1)!)^{2}}{(2(n+1))!}}{\frac{(n!)^{2}}{(2n)!}}\\
        &= \frac{(2n)!((n+1)(n!))^{2}}{(2n+2)(2n+1)(2n)!(n!)^{2}}\\
        &= \frac{((n+1)(n!))^{2}}{(2n+1)(2n+2)(n!)^{2}}\\
        &= \frac{(n+1)^{2}(n!)^{2}}{(4n^{2}+6n+2)(n!)^{2}}\\
        &= \frac{n^{2}+2n+1}{4n^{2}+6n+2}
        \shortintertext{This is less than 1, so series converges} 
\end{align*}

\end{document}
