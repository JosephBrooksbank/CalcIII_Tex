\documentclass[12pt]{article}

\usepackage{import}
\usepackage{xifthen}
\usepackage{pdfpages}
\usepackage{transparent}

\usepackage{geometry}
\usepackage{graphicx}
\usepackage{subcaption}
\usepackage{float}
\usepackage{amsmath}
\usepackage{mathtools}
\usepackage{fontspec}

\newcommand{\incfig}[1]{%
    \def\svgwidth{\columnwidth}
    \import{./figures/}{#1.pdf_tex}
}

% Title Page
\title{CALC Notes Day 27}
\author{Joseph Brooksbank}
\begin{document}
\maketitle

\section*{Written Homework Notes}
$y_n = \frac{2^{n}}{n!}$
Show $y_n$ $\to$ 0 with Squeeze Theorm:
\\
Trap $y_n$ between 2 seq $x_n, z_n$ \\ $x_n \leq y_n \leq z_n$ such that $x_n$ $\to$ 0, and $z_n$ $\to$

\fbox{\begin{minipage}{15em}
                IDEA: Write out $\frac{2^{n}}{n!} = \frac{2 * 2 * 2 * 2 (n times)}{n * (n-1) * (n-2) *...*2, 1}$
\end{minipage}}
What is that less than?
[
\begin{figure}[ht]
    \centering
    \incfig{fig_1}
    \caption{fig_1}
    \label{fig:fig_1}
\end{figure}

For the first one, it equals 2. However for every other term, they're less than or equal to 1, thus the entire thing is less than or equal to 2 
\\
EXCEPT for the last one, which is $\frac{2}{n}$, so the entire thing is less than $\frac{4}{n}$. Since $\frac{4}{n}$

\section*{Integral Test!}

\fbox{\begin{minipage}{30em}
If you have inf series $\sum_{n=1}^{\infty} x_n$ and \begin{enumerate}
\item $x_n$ can be "turned into function" 
\item $x_n$ is decreasing 
\item $x_n \geq 0$
\end{enumerate}
Then the convergent status of $\sum_{n=1}^{\infty} x_n $ is the same as $\int_{1}^{\infty} f(x_) dx $ "Can use integral to decide if series converges or diverges"
\end{minipage}}
However, this only actually helps if you can \textbf{ACTUALLY DO THE INTEGRAL} 

\subsection*{EX}
Does 
\[
\sum_{n=3}^{\infty} \frac{ln n}{n}
.\] Converge or diverge?

\begin{enumerate}
        \item f(x) = $\frac{ln x}{x}$ 
        \item is above 0, since x $\geq$ 3 
        \item Decreasing? $(\frac{ln x}{x})^{'} = \frac{\frac{1}{x}* x -ln x * 1}{x^{2}} = \frac{1-ln x}{x^{2}}$\\
                $ln x \geq 3$ \\
                ln 3 = 1.1 \\
                so, ENTIRE derivative is less than or equal to 0, so function is decreasing. 
\end{enumerate}
These were the steps to say that we're ALLOWED to use the integral test -- not the actual test itself 
\\

\begin{align*}
        \int_{3}^{\infty} \frac{ln x}{x} dx 
        \shortintertext{Use U sub, u = ln(x) , du = $\frac{1}{x}dx$} 
        &= \int_{}^{} u du \\
        &= \frac{u^{2}}{2} \\
        &= \frac{(lnx)^{2}}{2} \text{from $\infty$ to 3} \\
        &= \lim_{t\to\infty} \frac{(lnt)^{2}}{2} - \frac{(ln3)^{2}}{2}\\
        &= \frac{\infty}{2} = \infty 
\end{align*}

\subsection*{EX 2}

\begin{align*}
        \text{p > 0} and \\
        \sum_{n=1}^{\infty} \frac{1}{n^{P}} 
        \shortintertext{Integral Test:} 
        f(x) &= \frac{1}{x^{p}} \shortintertext{since p is always greater than 0, this is always a real number greater than 0} \\
             \text{because n is inc, so $n^{p}$ is inc, so $\frac{1}{n^{p}}$ is dec} \\
\end{align*}
Now we use the intergral test: 
\begin{align*}
        \int_{1}^{\infty} \frac{1}{x^{p}} dx &= \[ \begin{cases}
                        convg & p > 1\\
                        divg & p \leq 1
        \end{cases}
        \]
\end{align*}

\subsection*{one last thing...}
\[
\sum_{n=1}^{\infty} \frac{1}{n^{3}} 
.\] 
Does this converge or diverge? \\ Converge, p is greater than 1, see above
\\
\fbox{\begin{minipage}{20em}
Could we estimate this to within 0.01?
\end{minipage}}
\\
Try adding up the first 10 or terms of the series until we're within 0.01 of the infinite series 

\fbox{\begin{minipage}{15em}
                Question: if I do \[
                \frac{1}{1^{3}} + \frac{1}{2^{3}} + \frac{1}{3^{3}} + ... + \frac{1}{N^{3}}
                .\] How close am I do the infinite series?
\end{minipage}}

\begin{figure}[ht]
    \centering
    \incfig{adding-up-partial-sums}
    \caption{Adding up partial sums}
    \label{fig:adding-up-partial-sums}
\end{figure}

\end{document}
